\chapter{Valoraciones finales}
Como hemos detallado en la presente memoria, una de las tareas más importantes que se deben llevar a cabo en el aprendizaje automático es la validación de resultados obtenidos por los algoritmos de aprendizaje. El método estándar más aceptado en la actualidad por los analistas de datos es el de la aplicación de test estadísticos sobre los experimentos, que, entre otras utilidades, apoyan la toma de decisiones, como por ejemplo la elección del algoritmo más adecuado.

La aplicación de test estadísticos se engloba en el ámbito de la Inferencia Estadística, que es la parte de la estadística que estudia cómo sacar conclusiones generales (sujetas a un determinado grado de fiabilidad o significancia) para toda la población a partir del estudio de una muestra. En este proyecto de fin de grado, se ha desarrollado una plataforma web con la que poder obtener conclusiones de los resultados obtenidos por diferentes algoritmos sobre distintos conjuntos de datos para determinar, por ejemplo, si los algoritmos podrían tener un rendimiento significativamente diferente, y por lo tanto no se podrían considerar iguales.

El objetivo de este Trabajo de Fin de Grado ha sido el desarrollo de una plataforma web para la validación de experimentación en aprendizaje automático y minería de datos. En concreto, el trabajo descrito en esta memoria se resume en los siguientes puntos:

\begin{itemize}
\item Se extendió una librería de test estadísticos: nonparametric.py, actualmente implementada en Python.
\item Se crearon servicios web basados en REST para facilitar el uso de los test y para poder realizar la subida y consulta de datos (pertenecientes a la aplicación de distintos algoritmos sobre problemas o conjuntos de datos).
\item Se desarrolló una interfaz web que hace uso de estos servicios y que muestra los resultados obtenidos.
\end{itemize}

El objetivo ha sido que el analista de datos pudiese introducir en la web los resultados obtenidos mediante experimentación, y seleccionase el test estadístico que desease utilizar para que, de forma automática, la plataforma le mostrase los resultados de la aplicación del test. Así, plataforma permite, de un modo fácil y centralizado, la validación de resultados mediante el uso de test estadísticos.

Como se ha ido viendo a lo largo de los distintos capítulos (en concreto en el capítulo \ref{valiprue} perteneciente a la validación y pruebas), todos los objetivos del proyecto se han realizado con éxito. La librería de test estadísticos consta de todos los test deseados (únicamente falta el test de comparaciones múltiples de Li, que fue sustituido en su lugar por el test de Finner, como se comentó en la sección \ref{plantemp} debido a la ocurrencia del riesgo \textbf{RPD-3}). Además, los servicios web proporcionan el acceso a los test, tal y como se había planeado. Por otra parte, interfaz web cumple también con las historias de usuario vistas a lo largo del capítulo \ref{analisisreq}.

\section{Posibles mejoras}

A continuación se indican algunas ampliaciones con las que se podría extender la aplicación desarrollada:

\subsection{Test estadísticos para datos no apareados}
Uno de los puntos a destacar sobre el presente trabajo es que los test estadísticos desarrollados en este caso sirven para ser aplicados sobre un tipo de datos en particular: datos apareados. Disponemos de este tipo de datos cuando los datos de las muestras pertenecen a los mismos individuos (p. ej. aplicamos dos algoritmos sobre los mismos N conjuntos de datos, obteniendo dos muestras de tamaño N apareadas). Se podría extender la funcionalidad de la plataforma STAC permitiendo la aplicación de test estadísticos para datos no apareados. Existen actualmente test para este propósito. Por ejemplo, la contraposición para datos no apareados del test de los Rangos Signados de Wilcoxon presente en este proyecto sería el test U de Mann-Whitney.

\subsection{Datos de entrada}
Otra de las posibles mejoras es poder permitir al usuario la subida de datos no sólo mediante ficheros, sino a través de entrada manual de datos. Además, sería interesante poder permitir al usuario establecer él mismo el formato mediante el cual quiera subir los datos (p. ej. establecer el separador de columnas, el separador de filas, etc.). También se podría disponer de modificación vía web de los datos, tanto si son subidos mediante fichero de entrada o mediante formulario.

\subsection{Otras mejoras}
Se podrían añadir cambios a la interfaz con el fin de hacerla más atractiva al usuario. Por ejemplo, se podría mostrar una barra de progreso en la subida de datos. También se podría dar la posibilidad de mostrar los resultados con un determinado número de decimales (establecido por el usuario).