%***************************************************************************************************************************

\chapter{Introducción}
En el contexto tecnológico actual, en donde el Big Data es un recurso cada vez más utilizado, el rol
del analista de datos (data scientist) se está convirtiendo en una profesión emergente y de elevada
demanda. Un analista de datos es aquel profesional que reúne, analiza e interpreta los datos obtenidos
con el objetivo de sacar ciertas conclusiones de ellos y así tomar diferentes decisiones, con las que
aumentar la productividad en una organización. Relacionado con el analista de datos, un nuevo rol emergente
en muchas empresas es el de CDO (\textit{``Chief Data Officer"}). Este rol es el responsable de la gestión
y la utilización de la información como un activo para toda la empresa. El analista de datos combina diferentes
habilidades, especialmente las técnicas de la minería de datos y del aprendizaje automático (DM\&ML).

Según Mitchell \cite{mitchell}, una definición de aprendizaje automático sería la siguiente: un programa
de ordenador aprende a partir de una experiencia E a realizar una tarea T (de acuerdo con una medida de
rendimiento P), si su rendimiento al realizar T, medido con P, mejora gracias a la experiencia E. La
minería de datos, por otra parte, es un campo de las ciencias de la computación referido al proceso que trata
de descubrir patrones en grandes volúmenes de conjuntos de datos \cite{mineria}. Para ello utiliza, entre
otros métodos, técnicas estadísticas para deducir estos patrones y tendencias que existen en los datos. Por
lo general, estos patrones no pueden ser detectados mediante exploración tradicional debido a la complejidad o
la gran cantidad de datos.

Una de las tareas más importantes que se deben llevar a cabo en el aprendizaje automático es la
validación de resultados obtenidos por los algoritmos de aprendizaje. El método estándar más aceptado
en la actualidad es el de la aplicación de test estadísticos sobre los experimentos, que, entre otras
utilidades, apoyan la toma de decisiones, como por ejemplo la elección del algoritmo más adecuado.

En este proyecto hemos creado y desarrollado una plataforma para asistir al analista de
datos en el proceso de validación de resultados. Para ello, se extendió una librería de test
estadísticos, se crearon servicios web para facilitar su consulta y se desarrolló una interfaz web
que hace uso de estos servicios. El objetivo es que el analista pueda introducir en la web los datos obtenidos
mediante experimentación y seleccionar el test estadístico que desee utilizar para que, de forma automática, la
plataforma muestre los resultados de la aplicación del test. Así, la plataforma permitirá de un modo fácil y
centralizado la validación de resultados mediante el uso de test estadísticos.

La herramienta se incorporará en la lista de aplicaciones disponibles a través de la web del
CiTIUS para su acceso. El impacto y difusión del resultado del proyecto tiene el potencial de ser amplio,
ya que en la actualidad no existe ninguna herramienta que centralice la aplicación de los test estadísticos de
mayor utilidad para la validación de algoritmos de aprendizaje automático y que, además, resulte fácil de usar.

%***************************************************************************************************************************

\section{Objetivos del proyecto}
El proyecto se centra en crear y desarrollar una plataforma web para asistir al analista de datos
en el proceso de validación de los resultados obtenidos de diferentes algoritmos de aprendizaje. Para ello,
habrá que realizar las siguientes tareas:
\begin{enumerate}
\item Completar y extender una librería de test estadísticos, actualmente implementada en el lenguaje Python.

La librería estará formada por test paramétricos, test para evaluar las condiciones de aplicación de los test
paramétricos (para la normalidad y homocedasticidad), así como test no paramétricos. Estos test se verán en
detalle a lo largo del capítulo \ref{contraste} en las secciones \ref{parametricos}, \ref{condiciones} y
\ref{no_parametricos} respectivamente. La librería a extender se denomina ``nonparametric.py", y los test de
normalidad y homocedasticidad, así como la prueba  $\mathcal{T}$ de Student se tomarán de la librería de estadística
de Python SciPy (scipy.stats). El listado de test para el proyecto es el siguientes:

- Normalidad: Shapiro-Wilk, D’Agostino–Pearson y Kolmogorov–Smirnov.

- Homocedasticidad: Levene.

- Paramétricos: t-test, ANOVA, Bonferroni.

- No paramétricos: Wilcoxon, Friedman, Iman-Davenport, Rangos Alineados de Friedman, Quade, Bonferroni-Dunn,
Holm, Finner, Hochberg, Li, Shaffer.

\item Crear los servicios web en Python, basados en REST, que hagan disponible el acceso a los
test estadísticos vía web.

Los servicios REST están basados en los métodos HTTP (POST y GET para este proyecto). Asimismo, las peticiones
a los servicios web de los test incluirán todos los datos necesarios (petición completa e independiente) para
que el servidor no tenga que mantener ningún estado para procesar la petición. Los datos a transmitir (datos del
analista, resultados obtenidos por los test) con REST se podrán transferir mediante XML, JavaScript Object
Notation (JSON), o ambos. Cada servicio dispondrá de varias URIs distintas en función de los parámetros para dar
mayor versatilidad a la API.

\item Desarrollar una interfaz web (HTML + JavaScript) para facilitar el uso de los test sobre los
datos introducidos por el analista de datos.

Las tecnologías que se emplearán para desarrollarla serán: HTML, JavaScript y CSS. Un requisito para el proyecto
es que el analista pueda aplicar los test de la forma más sencilla posible, por lo que se tendrá en cuenta este
aspecto en el desarrollo.
\end{enumerate}

%***************************************************************************************************************************

\section{Organización del documento}
La finalidad de este documento es la de presentar los desarrollos realizados para resolver correctamente los objetivos definidos, explicando para ello cada una de las partes que componen la plataforma web y las tareas realizadas a lo largo del proceso.
\begin{itemize}
\item En el \textit{\textbf{capítulo 2}} se explican los conceptos básicos manejados en el proyecto. Para ello, se realiza un análisis detallado del contraste de hipótesis, tratando los conceptos básicos con ejemplos. Además, se explica la finalidad y funcionamiento de cada uno de los test disponibles en la plataforma.
\item El \textit{\textbf{capítulo 3}} hace un análisis de los requisitos identificados en el proyecto, utilizando para ello las historias de usuario empleadas en la metodología del proyecto.
\item El \textit{\textbf{capítulo 4}} describe la gestión del proyecto. Incluye el análisis de riesgos, la metodología de desarrollo, la gestión de la configuración, la planificación temporal y el análisis de costes.
\item En el \textit{\textbf{capítulo 5}} se propone una arquitectura a alto nivel del sistema y se definen cada una
de las partes de las que constará, así como las herramientas de diseño y desarrollo utilizadas.
\item El \textit{\textbf{capítulo 6}} explica tanto el diseño como la implementación a bajo nivel de la arquitectura
propuesta.
\item El \textit{\textbf{capítulo 7}} se definen las pruebas establecidas para poder comprobar que la herramienta es válida y que los objetivos se han cumplido. Se detallan además los resultados de las mismas.
\item Por último, en el \textit{\textbf{capítulo 8}} se establecen aquellas conclusiones derivadas de la realización del
proyecto, además de una breve indicación de lo que podría ser mejorable o ampliable en un futuro.
\end{itemize}

%***************************************************************************************************************************