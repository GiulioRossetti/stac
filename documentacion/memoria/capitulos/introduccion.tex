%***************************************************************************************************************************

\chapter{Introducción}
En el contexto tecnológico actual, en donde el Big Data es un recurso cada vez más utilizado, el rol
del analista de datos (data scientist) se está convirtiendo en una profesión emergente y de elevada
demanda. Un analista de datos es aquel profesional que reúne, analiza e interpreta los datos obtenidos
con el objetivo de sacar ciertas conclusiones de ellos y así tomar diferentes decisiones, con las que
aumentar la productividad en una organización. Relacionado con el analista de datos, un nuevo rol emergente
en muchas empresas es el de CDO (\textit{``Chief Data Officer"}). Este rol es el responsable de la gestión
y la utilización de la información como un activo para toda la empresa. El analista de datos combina diferentes
habilidades, especialmente las técnicas de la minería de datos y del aprendizaje automático (DM\&ML).

Según Mitchell \cite{mitchell}, una definición de aprendizaje automático sería la siguiente: un programa
de ordenador aprende a partir de una experiencia E a realizar una tarea T (de acuerdo con una medida de
rendimiento P), si su rendimiento al realizar T, medido con P, mejora gracias a la experiencia E. La
minería de datos, por otra parte, es un campo de las ciencias de la computación referido al proceso que trata
de descubrir patrones en grandes volúmenes de conjuntos de datos \cite{mineria}. Para ello utiliza, entre
otros métodos, análisis matemático mediante la estadística para deducir estos patrones y tendencias que
existen en los datos. Por lo general, estos patrones no pueden ser detectados mediante exploración tradicional
debido a la complejidad o la gran cantidad de datos.

Una de las tareas más importantes que se deben llevar a cabo en el aprendizaje automático es la
validación de resultados obtenidos por los algoritmos de aprendizaje. El método estándar más aceptado
en la actualidad es el de la aplicación de tests estadísticos sobre los experimentos, que, entre otras
utilidades, apoyan la toma de decisiones (por ejemplo la elección del algoritmo más adecuado).

En este proyecto hemos creado y desarrollado una plataforma para asistir al analista de
datos en el proceso de validación de resultados. Para ello, se extendió una librería de tests
estadísticos, se crearon servicios web para facilitar su consulta y se desarrolló una interfaz web
que hace uso de estos servicios. El objetivo es que el analista pueda introducir en la web los datos obtenidos
mediante experimentación y seleccionar el test estadístico que desee utilizar para que, de forma automática, el
sistema muestre los resultados de la aplicación del test. Así, el sistema permitirá de un modo fácil y
centralizado la validación de resultados mediante el uso de tests estadísticos.

La herramienta se incorporará en la lista de aplicaciones disponibles a través de la web del
CiTIUS para su acceso. El impacto y difusión del resultado del proyecto tiene el potencial de ser amplio,
ya que en la actualidad no existe ninguna herramienta que centralice la aplicación de los test estadísticos de
mayor utilidad para la validación de aprendizaje automático y que resulte fácil de usar.

%***************************************************************************************************************************

\section{Objetivos del proyecto}
El proyecto se centra en crear y desarrollar una plataforma web para asistir al analista de datos
en el proceso de validación de resultados obtenidos de diferentes algoritmos de aprendizaje. Para ello,
habrá que realizar las siguientes tareas:
\begin{enumerate}
\item Completar y extender una librería de test estadísticos, actualmente implementada en el
lenguaje Python.
\item Estudiar y conocer las definiciones de los tests estadísticos de uso habitual en aprendizaje
automático.
\item Crear los servicios web en Python, basados en REST, que hagan disponible el acceso a los
tests estadísticos vía web.
\item Desarrollar una interfaz web (HTML + JavaScript) para facilitar el uso de los tests sobre los
datos introducidos por el analista de datos.
\end{enumerate}

%***************************************************************************************************************************

\section{Organización del documento}
Sección que incluirá la descripción de los distintos apartados del documento.

%***************************************************************************************************************************
